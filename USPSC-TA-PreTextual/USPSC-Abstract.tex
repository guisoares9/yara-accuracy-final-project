%% USPSC-Abstract.tex
%\autor{Silva, M. J.}
\begin{resumo}[Abstract]
 \begin{otherlanguage*}{english}
	\begin{flushleft} 
		\setlength{\absparsep}{0pt} % ajusta o espaçamento dos parágrafos do resumo		
 		\SingleSpacing  		\imprimirautorabr~~\textbf{\imprimirtitleabstract}.	\imprimirdata.  \pageref{LastPage}p. 
		%Substitua p. por f. quando utilizar oneside em \documentclass
		%\pageref{LastPage}f.
		\imprimirtipotrabalhoabs~-~\imprimirinstituicao, \imprimirlocal, 	\imprimirdata. 
 	\end{flushleft}

	\OnehalfSpacing 

	This work evaluates the accuracy of the Yara collaborative robotic platform for intracranial electrode placement in stereoelectroencephalography (SEEG) procedures. SEEG is a minimally invasive technique used in epilepsy surgery, requiring precise electrode implantation to localize epileptogenic zones. The study introduces custom-designed cranial phantoms that replicate human anatomical structures, enabling controlled experiments to assess the technical performance of the robotic system. Accuracy validation is performed in both in-lab mock operating room (mock-OR) and operating room (OR) environments. Results demonstrate that the Yara system achieves placement errors comparable to established robotic platforms, with average target point error (TPE) of 1.8 mm in conical target phantom tests, entry point error (EPE) of 1.3 mm and TPE of 3.07 mm in synthetic brain phantom in mock-OR, EPE of 3.1 mm and TPE 5.8 mm in OR evaluation. The findings highlight the potential of collaborative robotics to enhance precision and efficiency in SEEG procedures, supporting future clinical applications in epilepsy surgery.

	\vspace{\onelineskip}
 
   \noindent 
   \textbf{Keywords}: Robotic Neurosurgery, Epilepsy Surgery, Collaborative Robotics, Cranial Phantom
 \end{otherlanguage*}
\end{resumo}
