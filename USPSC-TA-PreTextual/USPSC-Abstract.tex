%% USPSC-Abstract.tex
%\autor{Silva, M. J.}
\begin{resumo}[Abstract]
 \begin{otherlanguage*}{english}
	\begin{flushleft} 
		\setlength{\absparsep}{0pt} % ajusta o espaçamento dos parágrafos do resumo		
 		\SingleSpacing  		\imprimirautorabr~~\textbf{\imprimirtitleabstract}.	\imprimirdata.  \pageref{LastPage}p. 
		%Substitua p. por f. quando utilizar oneside em \documentclass
		%\pageref{LastPage}f.
		\imprimirtipotrabalhoabs~-~\imprimirinstituicao, \imprimirlocal, 	\imprimirdata. 
 	\end{flushleft}
	\OnehalfSpacing 

        This research introduces a novel concept for a robotic system, Yara, specifically designed for Stereoelectroencephalography (SEEG) neurosurgery. The platform integrates collaborative robotic assistance, leveraging the precision of robotic systems, and is adaptable to various surgical environments. The study evaluates the system’s final accuracy by replicating a SEEG neurosurgery procedure on a 3D reconstructed patient. The patient’s head, cranium, and brain models were reproduced using different materials derived from tomography. The surgical procedure was simulated, starting from planning, proceeding through fiducial registration, and culminating with electrode placement. A post-operative tomography was performed to compute the error between the planned electrode trajectories and the final position of each entry and target point, providing a comprehensive assessment of the system’s accuracy and effectiveness in SEEG neurosurgery.
    
   \vspace{\onelineskip}
 
   \noindent 
   \textbf{Keywords}:
 \end{otherlanguage*}
\end{resumo}
