%% USPSC-Resumo.tex
\setlength{\absparsep}{18pt} % ajusta o espaçamento dos parágrafos do resumo

\begin{resumo}
	\begin{flushleft} 
			\setlength{\absparsep}{0pt} % ajusta o espaçamento da referência	
			\SingleSpacing 
			\imprimirautorabr~~\textbf{\imprimirtituloresumo}.	\imprimirdata. \pageref{LastPage}p. 

			Este trabalho avalia a exatidão da plataforma robótica colaborativa Yara para posicionamento de eletrodos intracranianos em procedimentos de estereoeletroencefalografia (SEEG). SEEG é uma técnica minimamente invasiva usada em cirurgia de epilepsia, exigindo implantação precisa de eletrodos para localizar zonas epileptogênicas. O estudo apresenta \textit{phantom} cranianos que replicam estruturas anatômicas humanas, permitindo experimentos controlados para avaliar o desempenho técnico do sistema robótico. A metodologia abrange monitoramento de pacientes, fusão de imagens multimodais, planejamento cirúrgico e registro usando pontos anatômicos fiduciais. A validação da precisão é realizada em ambientes de laboratório e centro cirúrgico, considerando fontes intrínsecas e extrínsecas de erro. Os resultados demonstram que o sistema Yara atinge erros de posicionamento comparáveis ​​às plataformas robóticas estabelecidas, com erros médios de ponto-alvo de 1,8 mm em testes de phantom e 5,8 mm em simulações em centro cirúrgico. Os resultados destacam o potencial da robótica colaborativa para aumentar a precisão e a eficiência em procedimentos SEEG, apoiando futuras aplicações clínicas em cirurgia de epilepsia.

			\imprimirtipotrabalho~-~\imprimirinstituicao, \imprimirlocal, \imprimirdata. 
 	\end{flushleft}
\OnehalfSpacing 

\textbf{Palavras-chave}: Robotic Neurosurgery, Epilepsy Surgery, Collaborative Robotics, Cranial Phantom.
\end{resumo}