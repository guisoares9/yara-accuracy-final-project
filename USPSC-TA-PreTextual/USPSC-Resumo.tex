%% USPSC-Resumo.tex
\setlength{\absparsep}{18pt} % ajusta o espaçamento dos parágrafos do resumo

\begin{resumo}
	\begin{flushleft} 
			\setlength{\absparsep}{0pt} % ajusta o espaçamento da referência	
			\SingleSpacing 
			\imprimirautorabr~~\textbf{\imprimirtituloresumo}.	\imprimirdata. \pageref{LastPage}p. 
			%Substitua p. por f. quando utilizar oneside em \documentclass
			%\pageref{LastPage}f.
			\imprimirtipotrabalho~-~\imprimirinstituicao, \imprimirlocal, \imprimirdata. 
 	\end{flushleft}
\OnehalfSpacing 
			
% O laboratório AeroTech da EESC-USP iniciou um projeto de Robótica Colaborativa aplicada à Neurocirurgia e Neuro-navegação, visando melhorar o desenvolvimento científico-tecnológico das neurocirurgias. O objetivo principal é desenvolver um sistema robótico que colabore com neurocirurgiões durante os procedimentos cirúrgicos. Inspirado no sistema de estereoeletroencefalografia (SEEG), esse sistema permite a inserção de eletrodos no cérebro para melhor análise temporal do foco das crises epilépticas. 

% Porém, como a pesquisa necessita da introdução de um manipulador robótico dentro de um ambiente controlado, sala de cirurgia, a trajetória do manipulador deve ser segura. E como ele opera em um espaço de trabalho limitado \cite{Corke}, e conhecemos todas as suas dimensões, é desenvolvido nesse projeto um sistema de trajetória voltado para esse ambiente, levando em conta a possibilidade de contato com o paciente e os profissionais de saúde.

% A pesquisa resultou na criação de um sistema que prediz o movimento que o manipulador robótico e verifica antes de executá-lo se entrará dentro de uma região crítica, sendo essa ao redor da cabeça do paciente, e indica ao operador para tomar devidas ações que evitarão o contato. Essa capacidade de prever o movimento do robô durante a cirurgia promete aumentar significativamente a segurança dos procedimentos, abrindo caminho para futuros avanços na colaboração entre humanos e robôs nesse campo.

% Esta pesquisa concluiu um método que será altamente benéfico para médicos e cirurgiões, embora funcione apenas como auxílio. O sistema atualmente em uso apresenta algumas áreas onde o contato não é detectado automaticamente, exigindo, portanto, uma verificação por meio de sua ferramenta auxiliar, a animação (ver seção \ref{animation}). Essa animação fornece uma visualização antecipada dos movimentos do manipulador, permitindo ao profissional avaliar a possibilidade de contato.

% \textbf{Palavras-chave}: Cirurgia Assistiva. Robótica Médica. Comunicação em Cirurgia. Robôs Cirúrgicos. Cinemática Robótica. Dissertação. Trabalho de conclusão de curso (TCC). Animação.
\end{resumo}