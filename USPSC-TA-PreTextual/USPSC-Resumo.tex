%% USPSC-Resumo.tex
\setlength{\absparsep}{18pt} % ajusta o espaçamento dos parágrafos do resumo

\begin{resumo}
	\begin{flushleft} 
		\setlength{\absparsep}{0pt} % ajusta o espaçamento da referência	
		\SingleSpacing 
		\imprimirautorabr~~\textbf{\imprimirtituloresumo}.	\imprimirdata. \pageref{LastPage}p. 
	\end{flushleft}

	\OnehalfSpacing 

	Este trabalho avalia a exatidão da plataforma robótica colaborativa Yara para posicionamento de eletrodos intracranianos em procedimentos de estereoeletroencefalografia (SEEG). SEEG é uma técnica minimamente invasiva usada em cirurgia de epilepsia, exigindo implantação precisa de eletrodos para localizar zonas epileptogênicas. O estudo apresenta \textit{phantom} cranianos que replicam estruturas anatômicas humanas, permitindo experimentos controlados para avaliar o desempenho técnico do sistema robótico. A validação da precisão é realizada em ambientes de laboratório e centro cirúrgico. Os resultados demonstram que o sistema Yara atinge erros de posicionamento comparáveis ​​aos de plataformas robóticas estabelecidas, com erro médio no ponto-alvo (TPE) de 1,8 mm em testes com \textit{phantom} com alvos cônicos, erro no ponto de entrada (EPE) de 1,3 mm e TPE de 3,07 mm em \textit{phantom} com cérebro sintético em ambiente simulado (SE), EPE de 3,1 mm e TPE de 5,8 mm em testes em centro cirúrgico. Os resultados destacam o potencial da robótica colaborativa e a eficiência em procedimentos de SEEG, apoiando futuras aplicações clínicas em cirurgia de epilepsia.

	\OnehalfSpacing

	\imprimirtipotrabalho~-~\imprimirinstituicao, \imprimirlocal, \imprimirdata. 
\OnehalfSpacing 

\textbf{Palavras-chave}: Robotic Neurosurgery, Epilepsy Surgery, Collaborative Robotics, Cranial Phantom.
\end{resumo}